\chapter{Profesionalismo, ética y toma de decisiones}

\section{Profesionalismo, ética y responsabilidad legal}
El profesionalismo en ingeniería de software integra competencia técnica, conducta ética y cumplimiento legal. La práctica responsable exige no solo saber cómo diseñar y construir software, sino también entender el impacto social y legal de nuestras decisiones profesionales. En este sentido, \textcite{swebok2024} enfatiza que los códigos de conducta y los marcos regulatorios son elementos centrales para garantizar la confianza pública en los sistemas de software.

\begin{longtable}{p{3cm} p{12cm}}
\caption{Dimensiones del profesionalismo en ingeniería de software} \\
\toprule
\textbf{Dimensión} & \textbf{Descripción} \\
\midrule
\endfirsthead

\toprule
\textbf{Dimensión} & \textbf{Descripción} \\
\midrule
\endhead

\multicolumn{2}{r}{\textit{Continúa en la siguiente página}} \\
\endfoot

\bottomrule
\endlastfoot

Competencia técnica & Mantener y demostrar conocimientos actualizados (metodologías, lenguajes, prácticas de prueba y seguridad) y aplicarlos con juicio profesional. \\

Ética profesional & Actuar con integridad, transparencia y evitar conflictos de interés; reportar riesgos serios que puedan dañar a usuarios o terceros. \\

Responsabilidad legal & Cumplir leyes de propiedad intelectual, privacidad y regulaciones sectoriales; entender contractualidad y responsabilidad frente a fallos. \\

Impacto social & Evaluar consecuencias (sesgo, privacidad, seguridad) y diseñar mitigaciones para minimizar daños sociales. \\
\end{longtable}

\section{Dinámicas de grupo y habilidades de comunicación}
La ingeniería de software es esencialmente una actividad colectiva. El rendimiento del proyecto depende tanto de habilidades sociales (comunicación, liderazgo) como de capacidades técnicas.

\begin{longtable}{p{4cm} p{11cm}}
\caption{Dinámicas de grupo y su efecto en proyectos de software} \\
\toprule
\textbf{Dinámica} & \textbf{Efecto / uso en proyectos} \\
\midrule
\endfirsthead

\toprule
\textbf{Dinámica} & \textbf{Efecto / uso en proyectos} \\
\midrule
\endhead

\multicolumn{2}{r}{\textit{Continúa en la siguiente página}} \\
\endfoot

\bottomrule
\endlastfoot

Cooperativa & Facilita prácticas ágiles: planificación conjunta, revisiones de código y retroalimentación continua; reduce errores de integración y mejora la calidad. \\

Competitiva & Útil para innovación puntual (hackatones, concursos), pero si se mantiene sostenida puede fragmentar conocimiento y dificultar la colaboración. \\

Mixta & En organizaciones grandes, equipos colaboran internamente pero compiten por recursos; requiere coordinación, gobernanza y canales claros de transferencia de conocimiento. \\
\end{longtable}

\subsection*{Habilidades clave}
\begin{itemize}
  \item \textbf{Comunicación escrita:} documentación clara y concisa adaptada a distintos públicos.
  \item \textbf{Escucha activa:} comprender necesidades del cliente y del equipo para evitar malentendidos.
  \item \textbf{Presentación de resultados:} traducir lo técnico a lo no técnico y viceversa.
  \item \textbf{Resolución de conflictos:} negociar prioridades y compromisos sin sacrificar requisitos críticos.
\end{itemize}

\section{Fundamentos de economía y toma de decisiones}
Tomar decisiones informadas exige medir costos, beneficios y riesgos. El análisis económico aplicado al software ayuda a priorizar iniciativas y defender inversiones ante stakeholders.

\subsection*{Métricas económicas básicas}
\begin{itemize}
  \item \textbf{Costo total de propiedad (TCO):} suma de costos de desarrollo, operación y mantenimiento durante el ciclo de vida.
  \item \textbf{Retorno sobre la inversión (ROI):}
\end{itemize}

\begin{equation}\label{eq:roi}
ROI \;=\; \dfrac{\text{Beneficios proyectados} - \text{Costos de inversión}}{\text{Costos de inversión}}
\end{equation}
\noindent\textit{Ecuación \ref{eq:roi}. Cálculo del retorno sobre la inversión (ROI).}

\begin{itemize}
  \item \textbf{Valor Presente Neto (VPN):} evaluación de flujos de caja futuros descontados para decisiones multi-período.
\end{itemize}

\subsection*{Criterios de decisión}
Cuando hay múltiples alternativas, emplea criterios claros (valor esperado, ROI, TCO, impacto en plazo). Combinar métricas cuantitativas con juicios cualitativos (riesgo reputacional, cumplimiento legal) es práctica recomendable en proyectos críticos.

\section{Análisis de riesgo e incertidumbre}
Los riesgos en software pueden ser técnicos, de gestión o externos. Cuantificarlos y priorizarlos permite aplicar mitigaciones coste-efectivas.

\begin{longtable}{p{4cm} p{11cm}}
\caption{Clasificación breve de riesgos en proyectos de software} \\
\toprule
\textbf{Tipo de riesgo} & \textbf{Descripción / Ejemplo} \\
\midrule
\endfirsthead

\toprule
\textbf{Tipo de riesgo} & \textbf{Descripción / Ejemplo} \\
\midrule
\endhead

\multicolumn{2}{r}{\textit{Continúa en la siguiente página}} \\
\endfoot

\bottomrule
\endlastfoot

Técnico & Integración de tecnologías nuevas, deuda técnica que impide entregas puntuales, falta de pruebas automatizadas. \\

De gestión & Estimaciones erróneas, rotación de personal, dependencias externas incumplidas. \\

Externo & Cambios regulatorios, variaciones en el mercado, fallos de proveedores críticos. \\
\end{longtable}

\subsection*{Cuantificación rápida}
Un enfoque útil para priorizar riesgos es el Valor Esperado (EMV, Expected Monetary Value):

\begin{equation}\label{eq:emv}
EMV \;=\; \sum_{i} p_i \cdot c_i
\end{equation}
\noindent\textit{Ecuación \ref{eq:emv}. Valor Esperado (EMV) — probabilidad por impacto monetario para priorización de riesgos.}

La aplicación práctica del EMV requiere estimaciones conservadoras y revisión continua; sus fundamentos probabilísticos se apoyan en la literatura de estadística aplicada \parencite{ross2014}.

\subsection*{Estrategias de mitigación}
\begin{itemize}
  \item \textbf{Evitar:} cambiar el plan para eliminar riesgo crítico.
  \item \textbf{Transferir:} contratos, seguros y acuerdos SLA con penalidades claras.
  \item \textbf{Mitigar:} prototipos, pruebas tempranas, arquitecturas desacopladas y revisiones técnicas.
  \item \textbf{Aceptar:} asumir el riesgo cuando el costo de mitigación supera su impacto esperado.
\end{itemize}

\section{Análisis de compensación (trade-off analysis)}
El análisis de trade-offs permite comparar alternativas cuando los criterios son contradictorios (por ejemplo: rendimiento vs. costo, tiempo vs. calidad). Debe apoyarse en métricas medibles y, cuando proceda, en experimentación controlada.

\begin{longtable}{p{3cm} p{4cm} p{4cm} p{4cm}}
\caption{Ejemplo simplificado de análisis de compensación en arquitectura} \\
\toprule
\textbf{Alternativa} & \textbf{Ventaja principal} & \textbf{Desventaja} & \textbf{Adecuado para} \\
\midrule
\endfirsthead

\toprule
\textbf{Alternativa} & \textbf{Ventaja principal} & \textbf{Desventaja} & \textbf{Adecuado para} \\
\midrule
\endhead

\multicolumn{4}{r}{\textit{Continúa en la siguiente página}} \\
\endfoot

\bottomrule
\endlastfoot

Monolito & Simplicidad de despliegue y menor sobrecarga operacional & Escalabilidad y despliegue restringidos; acoplamiento elevado entre módulos & Proyectos pequeños / POC \\

Microservicios & Escalabilidad independiente y despliegue por servicio & Complejidad en orquestación, observabilidad e integración & Sistemas grandes, equipos distribuidos \\

Serverless & Baja inversión inicial y facturación por uso & Dependencia del proveedor y límites de ejecución & Prototipos, cargas variables y validaciones rápidas \\
\end{longtable}

\subsection*{Notas sobre trade-offs algorítmicos}
En elecciones algorítmicas o de diseño interno, la literatura clásica enfatiza que mejorar una métrica normalmente perjudica otra (por ejemplo, tiempo vs. espacio). Para decisiones de este tipo, apóyate en análisis de complejidad y medidas empíricas antes de optar por una solución \parencite{clrs2009,knuth1997}.
