\chapter*{CONCLUSIONES}
\label{cap:conclusiones}
\addcontentsline{toc}{chapter}{Conclusiones}

A lo largo de este trabajo se ha explorado la ingeniería de software a través de la perspectiva del SWEBOK, entendido como una guía que orienta la construcción de sistemas confiables y sostenibles. Más que un compendio de prácticas, constituye una brújula que acompaña el ciclo de vida del software, desde la concepción de una idea hasta el mantenimiento que asegura su vigencia en el tiempo.

El análisis de requisitos resaltó la necesidad de un entendimiento claro con los interesados para evitar inconsistencias en el producto final. El diseño arquitectónico reafirmó la importancia de una base sólida y flexible, capaz de adaptarse a cambios futuros. La construcción y codificación mostraron que el código no solo debe ser funcional, sino también comprensible y mantenible.

Las pruebas fueron concebidas como un mecanismo sistemático para identificar fallos y garantizar la calidad, mientras que el mantenimiento se presentó como un proceso continuo y fundamental para la evolución del software, considerando sus dimensiones correctiva, adaptativa, perfectiva y preventiva.

En los procesos, la gestión de configuración se destacó como pilar para el control de versiones y la trazabilidad. La planificación y el seguimiento se identificaron como prácticas esenciales para alinear esfuerzos, mientras que los modelos de ciclo de vida, tanto predictivos como adaptativos, evidenciaron la necesidad de ajustar la estrategia a las características del proyecto. El aseguramiento de la calidad y la medición se establecieron como bases para la mejora continua.

El estudio también subrayó el rol del profesionalismo y la ética en la toma de decisiones, considerando riesgos y compensaciones necesarias para mantener la responsabilidad en la práctica. Asimismo, los fundamentos técnicos en algoritmos, lógica, estadística, hardware y redes fueron reconocidos como el soporte esencial que posibilita la implementación de las prácticas descritas.

En síntesis, el SWEBOK no debe interpretarse únicamente como un marco de referencia, sino como una invitación a aplicar sus principios en proyectos reales, fomentando la construcción de software eficiente, sostenible y valioso para la sociedad.
