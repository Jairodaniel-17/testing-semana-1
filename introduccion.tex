\chapter*{INTRODUCCIÓN}
\label{cap:introduccion}
\addcontentsline{toc}{chapter}{Introducción}

Imagina que estás al frente de un proyecto para construir un rascacielos. Tienes a los mejores arquitectos, ingenieros y constructores, pero cada uno habla un idioma diferente. Unos miden en metros, otros en pies. Algunos entienden "cimentación" de una manera, y otros de otra completamente distinta. El equipo trabaja duro, pero las piezas no encajan. La comunicación se rompe y los plazos se sienten como una meta inalcanzable.
Durante décadas, fue el desafío central de la ingeniería de software, una disciplina que, a diferencia de la construcción de edificios, trabaja con materiales invisibles: la lógica, los datos y la creatividad. ¿Cómo podemos asegurarnos de que estamos construyendo software sólido, fiable y útil, y no solo castillos digitales en el aire?
Aquí es donde entra en juego el SWEBOK. Piensa en el SWEBOK (Guide to the Software Engineering Body of Knowledge o Guía del Conocimiento de la Ingeniería de Software) no como un libro de recetas rígidas, sino más bien como un mapa del territorio. Es el resultado de décadas de experiencia, lecciones aprendidas (a menudo de la manera más difícil) y el conocimiento colectivo de miles de profesionales de todo el mundo, destilado en una guía accesible.
Su propósito no es decirte exactamente cómo escribir cada línea de código, sino darte algo mucho más valioso:
Un lenguaje común, para que cuando hablemos de "requisitos", "diseño" o "pruebas", todos estemos en la misma página.
Una estructura clara que organiza este vasto campo en áreas de conocimiento manejables, desde la gestión de la configuración hasta la calidad del software.
Un panorama completo que nos ayuda a ver cómo cada pieza del rompecabezas, desde la primera conversación con un cliente hasta el mantenimiento a largo plazo, se conecta con las demás.
